\documentclass{article}
\usepackage{amsmath}

\begin{document}
$A = \{0, 1, 2, 3, 4\}$  $B = \{0, 1, 2, 3\}$ \\
\begin{enumerate}
    \item[a\\)] $a = b$ \\ 
    $R = \{(0,0), (1,1), (2, 2), (3, 3)\}$
    \item[b\\)] $a + b = 4$ \\
    $a + b = 4 \Rightarrow b = 4 - a$
    substituindo cada valor que existe no conjunto a 
    verificando se existe um elemento em b que é igual
    ao resultado. \\ $R = \{(1, 3), (2, 2), (3, 1) (4, 0)\}$
    \item[c\\)] $a > b$ \\ 
    $R = \{(1,0), (2, 1), (2, 0), (3, 2), (3, 1), (3,0), (4, 3), (4, 2), (4, 1), (4, 0) \}$.
    \item[d\\)] $\text{mdc}(a, b) = 1$ \\
    O zero não vai entrar 
    nessa sequência, porque ele não possui divisor.\\
    $R = \{(1,1), (1,2), (1,3), (2, 1), (2, 3), (3, 1), 
    (3, 2) , (4, 1), (4, 3)\}$
    \item[e\\)] $a | b$ \\
    $R = \{(1,1), (1, 2), (1, 3), (2, 2), (2, 4), (3, 3)\}$
    \item[f\\)] $\text{mmc}(a, b) = 2$. \\
    $R = \{(1, 2), (2, 2), (2, 4)\}$
\end{enumerate}

\end{document}
